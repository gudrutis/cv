%% start of file `template.tex'.
%% Copyright 2006-2013 Xavier Danaux (xdanaux@gmail.com).
%
% This work may be distributed and/or modified under the
% conditions of the LaTeX Project Public License version 1.3c,
% available at http://www.latex-project.org/lppl/.


\documentclass[11pt,letterpaper,sans]{moderncv}\usepackage[]{graphicx}\usepackage[]{color}
%% maxwidth is the original width if it is less than linewidth
%% otherwise use linewidth (to make sure the graphics do not exceed the margin)
\makeatletter
\def\maxwidth{ %
  \ifdim\Gin@nat@width>\linewidth
    \linewidth
  \else
    \Gin@nat@width
  \fi
}
\makeatother

\definecolor{fgcolor}{rgb}{0.345, 0.345, 0.345}
\newcommand{\hlnum}[1]{\textcolor[rgb]{0.686,0.059,0.569}{#1}}%
\newcommand{\hlstr}[1]{\textcolor[rgb]{0.192,0.494,0.8}{#1}}%
\newcommand{\hlcom}[1]{\textcolor[rgb]{0.678,0.584,0.686}{\textit{#1}}}%
\newcommand{\hlopt}[1]{\textcolor[rgb]{0,0,0}{#1}}%
\newcommand{\hlstd}[1]{\textcolor[rgb]{0.345,0.345,0.345}{#1}}%
\newcommand{\hlkwa}[1]{\textcolor[rgb]{0.161,0.373,0.58}{\textbf{#1}}}%
\newcommand{\hlkwb}[1]{\textcolor[rgb]{0.69,0.353,0.396}{#1}}%
\newcommand{\hlkwc}[1]{\textcolor[rgb]{0.333,0.667,0.333}{#1}}%
\newcommand{\hlkwd}[1]{\textcolor[rgb]{0.737,0.353,0.396}{\textbf{#1}}}%

\usepackage{framed}
\makeatletter
\newenvironment{kframe}{%
 \def\at@end@of@kframe{}%
 \ifinner\ifhmode%
  \def\at@end@of@kframe{\end{minipage}}%
  \begin{minipage}{\columnwidth}%
 \fi\fi%
 \def\FrameCommand##1{\hskip\@totalleftmargin \hskip-\fboxsep
 \colorbox{shadecolor}{##1}\hskip-\fboxsep
     % There is no \\@totalrightmargin, so:
     \hskip-\linewidth \hskip-\@totalleftmargin \hskip\columnwidth}%
 \MakeFramed {\advance\hsize-\width
   \@totalleftmargin\z@ \linewidth\hsize
   \@setminipage}}%
 {\par\unskip\endMakeFramed%
 \at@end@of@kframe}
\makeatother

\definecolor{shadecolor}{rgb}{.97, .97, .97}
\definecolor{messagecolor}{rgb}{0, 0, 0}
\definecolor{warningcolor}{rgb}{1, 0, 1}
\definecolor{errorcolor}{rgb}{1, 0, 0}
\newenvironment{knitrout}{}{} % an empty environment to be redefined in TeX

\usepackage{alltt}        % possible options include font size ('10pt', '11pt' and '12pt'), paper size ('a4paper', 'letterpaper', 'a5paper', 'legalpaper', 'executivepaper' and 'landscape') and font family ('sans' and 'roman')

% moderncv themes
\moderncvstyle{casual}                             % style options are 'casual' (default), 'classic', 'oldstyle' and 'banking'
\moderncvcolor{blue}                               % color options 'blue' (default), 'orange', 'green', 'red', 'purple', 'grey' and 'black'
%\renewcommand{\familydefault}{\sfdefault}         % to set the default font; use '\sfdefault' for the default sans serif font, '\rmdefault' for the default roman one, or any tex font name
%\nopagenumbers{}                                  % uncomment to suppress automatic page numbering for CVs longer than one page

% character encoding
\usepackage[utf8]{inputenc}                       % if you are not using xelatex ou lualatex, replace by the encoding you are using
%\usepackage{CJKutf8}                              % if you need to use CJK to typeset your resume in Chinese, Japanese or Korean

% adjust the page margins
\usepackage[scale=0.75]{geometry}
%\setlength{\hintscolumnwidth}{3cm}                % if you want to change the width of the column with the dates
%\setlength{\makecvtitlenamewidth}{10cm}           % for the 'classic' style, if you want to force the width allocated to your name and avoid line breaks. be careful though, the length is normally calculated to avoid any overlap with your personal info; use this at your own typographical risks...

% personal data
\name{Leonardo}{Collado-Torres}
%\title{Resumé title}                               % optional, remove / comment the line if not wanted
\address{615 N. Wolfe Street, Room E3032}{21205-2179}{United States}% optional, remove / comment the line if not wanted; the "postcode city" and and "country" arguments can be omitted or provided empty
%\phone[mobile]{+1~(234)~567~890}                   % optional, remove / comment the line if not wanted
\phone[fixed]{+1~(410)~955~0958}                    % optional, remove / comment the line if not wanted
%\phone[fax]{+3~(456)~789~012}                      % optional, remove / comment the line if not wanted
\email{lcollado@jhu.edu}                               % optional, remove / comment the line if not wanted
\homepage{http://www.biostat.jhsph.edu/$\sim$lcollado/}                         % optional, remove / comment the line if not wanted
\extrainfo{Blog: http://lcolladotor.github.io/}                 % optional, remove / comment the line if not wanted
\photo[64pt][0.4pt]{picture}                       % optional, remove / comment the line if not wanted; '64pt' is the height the picture must be resized to, 0.4pt is the thickness of the frame around it (put it to 0pt for no frame) and 'picture' is the name of the picture file
%\quote{Some quote}                                 % optional, remove / comment the line if not wanted

% to show numerical labels in the bibliography (default is to show no labels); only useful if you make citations in your resume
%\makeatletter
%\renewcommand*{\bibliographyitemlabel}{\@biblabel{\arabic{enumiv}}}
%\makeatother
%\renewcommand*{\bibliographyitemlabel}{[\arabic{enumiv}]}% CONSIDER REPLACING THE ABOVE BY THIS

% bibliography with mutiple entries
%\usepackage{multibib}
%\newcites{book,misc}{{Books},{Others}}
%----------------------------------------------------------------------------------
%            content
%----------------------------------------------------------------------------------
\IfFileExists{upquote.sty}{\usepackage{upquote}}{}
\begin{document}
%\begin{CJK*}{UTF8}{gbsn}                          % to typeset your resume in Chinese using CJK
%-----       resume       ---------------------------------------------------------
\makecvtitle

\section{Education}
\cventry{2011-present}{PhD in Biostatistics}{Johns Hopkins Bloomberg School of Public Health}{Baltimore, US}{}{} 
\cventry{2005-2009}{Bachelor in Genomic Sciences (LCG)}{National Autonomous University of Mexico (UNAM)}{Cuernavaca, MX}{\textit{Grade 9.71/10}}{}  % arguments 3 to 6 can be left empty
\cventry{2002-2005}{High school}{ITESM Campus Cuernavaca}{Cuernavaca, MX}{\textit{Grade 97.8/100}}{}

\section{PhD thesis}
\cvitem{title}{\emph{Developing single-base RNA-seq methods and software to understand neuropsychiatric disorders and more}.}
\cvitem{advisors}{\textbf{Jeffrey T. Leek} and \textbf{Andrew E. Jaffe}.}
\cvitem{description}{The goal is to develop statistical methods and software that enable researchers to differentiate the sources of variation observed in RNA-seq while minimizing the dependance on known annotation. This will allow researchers to correct for technological variation and study the biological variation driving their phenotype of interest. Then apply these methods to further our understanding of neuropsychiatric disorders using the Lieber Institute for Brain Development human brains collection ($>$ 1000 samples).}

\section{Honors and awards}
\cvitem{2011}{Awarded CONACyT Mexico scholarship for PhD studies outside Mexico.}
\cvitem{2009}{Honorable Mention for bachelor studies at LCG-UNAM.}
\cvitem{2005}{Best high school average: awarded ITESM system 90\% scholarship for college studies, declined to join LCG-UNAM.}

\section{Experience}
\subsection{Vocational}
\cventry{2009-2011}{Scientific executive}{Winter Genomics}{Cuernavaca, MX}{}{Responsible for recruiting and hiring new personnel, overseeing and supervising nine bioinformaticians, training new employees, writing research reports and presenting them to colleagues, and organizing all scientific projects.\newline{}%
\begin{itemize}%
\item First scientific staff member at Winter Genomics;
\item Projects completed:
  \begin{itemize}%
  \item de novo genome assembly simulations,
  \item assembly and annotation of the \emph{phiVC8} bacteriophage genome,
  \item de novo assembly of four \emph{Escherichia coli} strains and lead to Aguilar et al, PMID 22884033;
  \end{itemize}
\item Designed training material for new employees.
\end{itemize}}
\subsection{Research}
\cventry{2009--2011}{Bioinformatician}{Enrique Morett lab}{IBT-UNAM, Cuernavaca, MX}{}{Identified transcriptions start sites and transcription units in \emph{Escherichia coli} and \emph{Geobacter sulfurreducens} with RNA-seq data.\newline{}Developed the \emph{BacterialTranscription} R package.}
\cventry{2007--2009}{Undergraduate research assistant}{Guillermo D\'avila lab}{CCG-UNAM, Cuernavaca, MX}{}{Determined bacteriophage ecological groups by developing a method based on codon distribution of all phage sequenced genomes.\newline{}Joint work with Sur Herrera Paredes.}
\cventry{2007}{Undergraduate research assistant}{Roberto Kolter lab}{Harvard, Boston, US}{}{Supervisor: Elizabeth Shank.\newline{}Carried out screenings to identify bacteria that activate the production of exopolysaccharide through the activation of the gene tasA in \emph{Bacillus subtilis}.}

\section{Publications}
\subsection{Peer-reviewed}
    \begin{enumerate}
        \item Wright C, Shin JH, Rajpurohit A, Deep-Soboslay A, \textbf{Collado-Torres L}, Brandon NJ, Hyde TM, Kleinman JE, Jaffe AE, Cross AJ, Weinberger DR. Altered expression of histamine signaling genes in autism spectrum disorder. \emph{Translational Psychiatry} 2017. doi: \httplink[10.1038/tp.2017.87]{doi.org/10.1038/tp.2017.87}.
        
        \item \textbf{Collado-Torres L}$^{*}$, Nellore A$^{*}$, Kammers K, Ellis SE, Taub MA, Hansen KD, Jaffe AE, Langmead B, Leek JT. Reproducible RNA-seq analysis using \emph{recount2}. \emph{Nature Biotechnology} 2017. doi: \httplink[10.1038/nbt.3838]{doi.org/10.1038/nbt.3838}.
        \\ Pre-print: \emph{bioRxiv} 068478 (2016). doi: \httplink[10.1101/068478]{doi.org/10.1101/068478}.
        
        \item Nellore A, Jaffe AE, Fortin JP, Alquicira-Hernández J, \textbf{Collado-Torres L}, Wang S, Phillips RA, Karbhari N, Hansen KD, Langmead B, Leek JT. Human splicing diversity and the extent of unannotated splice junctions across human RNA-seq samples on the Sequence Read Archive. \emph{Genome Biology} 2016. doi: \httplink[10.1186/s13059-016-1118-6]{doi.org/10.1186/s13059-016-1118-6}.
        \\ Pre-print: \emph{bioRxiv} 038224 (2016). doi: \httplink[10.1101/038224]{doi.org/10.1101/038224}.
        
        \item \textbf{Collado-Torres L}, Nellore A, Frazee AC, Wilks C, Love MI, Langmead B, Irizarry RA, Leek JT, Jaffe AE. Flexible expressed region analysis for RNA-seq with derfinder. \emph{Nucl. Acids Res.} 2016. doi: \httplink[10.1093/nar/gkw852]{doi.org/10.1093/nar/gkw852}.
        \\ Pre-print: \emph{bioRxiv} 015370 (2016). doi: \httplink[10.1101/015370]{doi.org/10.1101/015370}.
        
        \item Nellore A, \textbf{Collado-Torres L}, Jaffe AE, Alquicira-Hernández J, Wilks C, Pritt J, Morton J, Leek JT, Langmead B. Rail-RNA: Scalable analysis of RNA-seq splicing and coverage. \emph{Bioinformatics} 2016. doi: \httplink[10.1093/bioinformatics/btw575]{doi.org/10.1093/bioinformatics/btw575}.
        \\ Pre-print: \emph{bioRxiv} 019067 (2015). doi: \httplink[10.1101/019067]{doi.org/10.1101/019067}.
        
        \item \textbf{Collado-Torres L}, Jaffe AE and Leek JT. regionReport: Interactive reports for region-level and feature-level genomic analyses [version2; referees: 2 approved, 1 approved with reservations]. \emph{F1000Research} 2016, 4:105. doi: \httplink[10.12688/f1000research.6379.2]{doi.org/10.12688/f1000research.6379.2}.
        \\ Pre-print: \emph{bioRxiv} 016659 (2015). doi: \httplink[10.1101/016659]{doi.org/10.1101/016659}.
        
        \item Jaffe AE, Shin J, \textbf{Collado-Torres L}, Leek JT, et al. Developmental regulation of human cortex transcription and its clinical relevance at single base resolution. \emph{Nat. Neurosci.} 2015. doi: \httplink[10.1038/nn.3898]{doi.org/10.1038/nn.3898}.
        
        \item Shank EA, Klepac-Ceraj V, \textbf{Collado-Torres L}, Powers GE, Losick R, Kolter R. Interspecies interactions that result in Bacillus subtilis forming biofilms are mediated mainly by members of its own genus. \emph{Proc. Natl. Acad. Sci.} U.S.A. 2011 Nov;108(48):E1236–1243. doi: \httplink[10.1073/pnas.1103630108]{doi.org/10.1073/pnas.1103630108}.
        
        \item Gama-Castro S, Salgado H, Peralta-Gil M, Santos-Zavaleta A, Muñiz-Rascado L, Solano-Lira H, Jimenez-Jacinto V, Weiss V, Garc\'ia-Sotelo JS, L\'opez-Fuentes A, Porr\'on-Sotelo L, Alquicira-Hern\'andez S, Medina-Rivera A, Mart\'inez-Flores I, Alquicira-Hern\'andez K, Mart\'inez-Adame R, Bonavides-Mart\'inez C, Miranda-R\'ios J, Huerta AM, Mendoza-Vargas A, \textbf{Collado-Torres L}, Taboada B, Vega-Alvarado L, Olvera M, Olvera L, Grande R, Morett E, Collado-Vides J. RegulonDB version 7.0: transcriptional regulation of Escherichia coli K-12 integrated within genetic sensory response units (Gensor Units). \emph{Nucleic Acids Res.} 2011 Jan;39(Database issue):D98–105. doi: \httplink[10.1093/nar/gkq1110]{doi.org/10.1093/nar/gkq1110}.
    \end{enumerate}
\subsection{Pre-prints}
    \begin{enumerate}
        \item Jaffe AE, Straub R, Shin JH, Tao R, Gao Y, \textbf{Collado-Torres L}, Kam-Thong T, Xi HS, Quan J, Chen Q, Colantuoni C, Ulrich B, Maher BJ, Deep-Soboslay A, The BrainSeq Consortium, Cross A, Brandon NJ, Leek JT, Hyde TM, Kleinman JE, Weinberger DR. Developmental and genetic regulation of the human cortex transcriptome in schizophrenia. \emph{bioRxiv} 124321 (2017). doi: \httplink[10.1101/124321]{doi.org/10.1101/124321}.
    \end{enumerate}
\subsection{Books}
    \begin{enumerate}
        \item Frazee AC, \textbf{Collado-Torres L}, Jaffe AE, Langmead B, Leek JT. Measurement, Summary, and Methodological Variation in RNA-sequencing in Statistical Analysis of Next Generation Sequencing Data, \emph{Springer}, 2014, 115-128.
    \end{enumerate}



\section{Referee for}
\cvitem{2013-2014}{Biostatistics}

\section{Presentations}
\subsection{Talks at conferences}
\cvitem{2016}{recount: facilitando el análisis de miles de muestras de RNA-seq, \emph{Genomeeting2016}, Mexico City -- MX.}
\cvitem{2016}{Using Data Science to Study Human Brain Genomic Measurements, \emph{SACNAS}, Long Beach -- US.}
\cvitem{2016}{\textbf{Collado-Torres L}, et al. Annotation-agnostic differential expression analysis, \emph{ENAR}, Austin -- US. \httplink[(slides)]{www.slideshare.net/lcolladotor/annotationagnostic-differential-expression-analysis}}
\cvitem{2015}{\textbf{Collado-Torres L}, Frazee AC, Love MI, Irizarry RA, Jaffe AE, Leek JT. Annotation-agnostic differential expression analysis, \emph{Genomics and Bioinformatics Symposium}, Center for Computational Genomics, Hopkins, Baltimore -- US.}
\cvitem{2015}{Jaffe AE, Shin J, \textbf{Collado-Torres L}, Leek JT, et al. Dissecting human brain development at high resolution using RNA-seq, \emph{ENAR}, Miami -- US. \httplink[(slides)]{www.slideshare.net/lcolladotor/dissecting-human-brain-development-at-high-resolution-using-rnaseq}}
\cvitem{2014}{Jaffe AE, Shin J, \textbf{Collado-Torres L}, Leek JT, et al. Developmental regulation of human cortex transcription at base-pair resolution, \emph{is3b}: 1st International Summer Symposium on Systems Biology, INMEGEN, Mexico City -- MX.}
\cvitem{2014}{\textbf{Collado-Torres L}, Frazee AC, Love MI, Irizarry RA, Jaffe AE, Leek JT. Fast differential expression analysis annotation-agnostic across groups with biological replicates, LCG 10 year anniversary, LCG-UNAM, Cuernavaca -- MX.}
\cvitem{2013}{\textbf{Collado-Torres L}, Frazee AC, Irizarry RA, Jaffe AE, Leek JT. Differential expression analysis of RNA-seq data at base-pair resolution in multiple biological replicates, \emph{useR2013}, Albacete -- Spain.}
\cvitem{2010}{\textbf{Collado-Torres L}, Reyes-Quiroz A, Cu\'ellar-Partida G, Moreno-Mayar V, Vargas-Ch\'avez C, Collado-Vides J. BacterialTranscription: a R package to identify Transcription Start Sites and Transcription Units, \emph{Bioconductor Developer Meeting}, EMBL, Heidelberg -- Germany.}

\subsection{Posters}
\cvitem{2015}{\textbf{Collado-Torres L}, Frazee AC, Love MI, Irizarry RA, Jaffe AE, Leek JT. Annotation-agnostic RNA-seq differential expression analysis software, \emph{ASHG2015} and \emph{IDIES2015}, Baltimore -- US.}
\cvitem{2014}{\textbf{Collado-Torres L}, Frazee AC, Love MI, Irizarry RA, Jaffe AE, Leek JT. Fast annotation-agnostic differential expression analysis, \emph{ENAR} and \emph{Delta Omega Poster Competition (JHBSPH)}, Baltimore -- US.}
\cvitem{2013}{\textbf{Collado-Torres L}, Jaffe AE, Leek JT. Fast annotation-agnostic differential expression analysis, \emph{Genomics and Bioinformatics Symposium}, Center for Computational Genomics, Hopkins, Baltimore -- US.}
\cvitem{2010}{\textbf{Collado-Torres L}, Reyes-Quiroz A, Cu\'ellar-Partida A, Moreno-Mayar V, Taboada B, Vega-Alvarado L, Jim\'enez-Jacinto V, Mendoza-Vargas A, Grande R, Olvera L, Olvera M, Vargas-Ch\'avez C, Júarez K, Collado-Vides J, Morett E. Global Analysis of Transcription Start Sites and Transcription Units in Bacterial Genomes, \emph{From Functional Genomics to Systems Biology}, EMBL, Heidelberg -- Germany.}
\cvitem{2010}{\textbf{Collado-Torres L}, Reyes-Quiroz A, Cu\'ellar-Partida A, Moreno-Mayar V, Taboada B, Vega-Alvarado L, Jim\'enez-Jacinto V, Mendoza-Vargas A, Grande R, Olvera L, Olvera M, Vargas-Ch\'avez C, Júarez K, Collado-Vides J, Morett E. Global Analysis of Transcription Start Sites and Transcription Units in Bacterial Genomes, \emph{BioC2010}, FHCRC, Seattle -- US.}

\subsection{Other talks}
\cvitem{2015}{dbFinder, \emph{Joint Genomic Meeting}, JHBSPH, Baltimore -- US.}
\cvitem{2015}{Easy parallel computing with BiocParallel and HTML reports with knitrBootstrap, \emph{Biostatistics Computing Club}, JHBSPH, Baltimore -- US.}
\cvitem{2015}{Does mapping simulated RNA-seq reads provide information?, \emph{Joint Genomic Meeting}, JHBSPH, Baltimore -- US.}
\cvitem{2014}{Git for research, \emph{Biostatistics Computing Club}, JHBSPH, Baltimore -- US.}
\cvitem{2013}{Introduction to ggbio, \emph{Genomics for Students}, JHBSPH, Baltimore -- US.}
\cvitem{2013}{Introduction to knitr, \emph{Biostatistics Computing Club}, JHBSPH, Baltimore -- US.}
\cvitem{2013}{Introduction to High-Throughput Sequencing and RNA-seq, \emph{Genomics for Students}, JHBSPH, Baltimore -- US.}
\cvitem{2012}{DEXSeq paper discussion, \emph{Genomics for Students}, JHBSPH, Baltimore -- US.}
\cvitem{2012}{Introduction to R and Biostatistics, LCG-UNAM via Skype.}
\cvitem{2012}{Introducing Git while making your academic webpage, \emph{Biostatistics Computing Club}, JHBSPH, Baltimore -- US.}
\cvitem{2011}{Introducing Biostatistics to first year LCG students, LCG-UNAM via Skype.}
\cvitem{2010}{Introduction to using Bioconductor for High Throughput Sequencing Analysis, \emph{National Bioinformatics Week}, CCG-UNAM, Cuernavaca -- MX.}
\cvitem{2009}{Bacteriophages: analyzing their diversity, \emph{LCG third generation symposium}, CCG-UNAM, Cuernavaca -- MX.}

\section{Teaching Experience}
\subsection{Instructor}

\begin{itemize}
\item PDCB-UNAM, Cuernavaca, MX

\cvitem{2011}{Invited instructor for the course \emph{Introduction to R and Biostatistics} \url{http://lcolladotor.github.io/courses/PDCB-Biostats.html} $\sim 10$ enrollment.} 
\cvitem{2010}{\emph{Analysis of High-Throughput Sequencing data with Bioconductor} for Biomedical Sciences PhD Program students \url{http://lcolladotor.github.io/courses/PDCB-HTS.html} $\sim10$ enrollment.}
 
\item CCG-UNAM, Cuernavaca, MX

\cvitem{2010}{\emph{Introduction to Using Bioconductor for High-Throughput Sequencing Analysis} practice lab at the \emph{National Bioinformatics Week} $\sim40$ enrollment.}

\item IBT-UNAM, Cuernavaca, MX

\cvitem{2010}{\emph{Introduction to R and plotting with R} course for Morett's lab $\sim10$ enrollment.}
\cvitem{2010}{Organized and gave a lecture for the course on \emph{Statistical Methods and Analysis of Genomic Data} \url{http://lcolladotor.github.io/courses/MEyAdDG.html} $\sim20$ enrollment.}
\cvitem{2009}{Organized the course \emph{Introduction to Bioinformatics} for Morett's lab and served as instructor for the \emph{Introduction to R and plotting with R} module \url{http://lcolladotor.github.io/courses/mIntroR.html} $\sim10$ enrollment.}

\item LCG-UNAM, Cuernavaca, MX

\cvitem{2009}{\emph{Seminar III: R/Bioconductor}. In-depth Bioconductor course \url{http://www.lcg.unam.mx/~lcollado/B/} $\sim30$ enrollment.}
\end{itemize}


\subsection{Guest lecturer}
\begin{itemize}
    
\item JHSPH, Baltimore, US
    
\cvitem{2015}{\emph{Introduction to R for Public Health Researchers: Reproducible research module} $\sim20$ enrollment.}

\item LCG-UNAM, Cuernavaca, MX

\cvitem{2012}{\emph{Introduction to R and Biostatistics} lecture for \emph{Seminar 1: Introduction to Bioinformatics} course $\sim30$ enrollment.}
\cvitem{2011}{\emph{Introduction to R and Biostatistics} lecture for \emph{Seminar 1: Introduction to Bioinformatics} course $\sim30$ enrollment.}
\end{itemize}


\subsection{Lead teaching assistant}
\begin{itemize}
    
\item JHSPH, Baltimore, US

\cvitem{2015--2016}{\emph{Statistical Methods in Public Health II} $\sim550$ enrollment.}    
\cvitem{2014--2015}{\emph{Statistical Methods in Public Health I} and \emph{II} $\sim550$ enrollment.}

\end{itemize}


\subsection{Teaching assistant}

\begin{itemize}
    
\item JHSPH, Baltimore, US

\cvitem{2014--2016}{\emph{MPH capstone project}: 30 min one-on-one consulting sessions (biostatistics, Stata coding) $\sim500$ enrollment. Develop and maintain \url{https://lcolladotor.shinyapps.io/MPHcapstoneTA/}.}
\cvitem{2015--2016}{\emph{Statistical Methods in Public Health I} $\sim550$ enrollment.}
\cvitem{2015}{\emph{Introduction to R for Public Health Researchers} $\sim20$ enrollment.}
\cvitem{2013--2014}{\emph{Statistical Methods in Public Health I} and \emph{II} $\sim550$ enrollment.}
\cvitem{2012--2013}{\emph{Statistical Methods in Public Health I, II, III,} and \emph{IV} $\sim550$ enrollment.}

    
\item LCG-UNAM, Cuernavaca, MX

\cvitem{2009}{\emph{Principles of Statistics}. Basic R \url{http://www.lcg.unam.mx/~lcollado/E/} $\sim30$ enrollment.}
\cvitem{2008}{\emph{Bioinformatics and Statistics I}. R and Bioconductor overview \url{http://www.lcg.unam.mx/~lcollado/R/} $\sim40$ enrollment.}
\end{itemize}

\section{Courses and Meetings Attendance}

\cvitem{2015}{\emph{ENAR}, Miami -- US.}
\cvitem{2014}{\emph{is3b}, INMEGEN, Mexico City -- MX.}
\cvitem{2014}{\emph{BioC2014}, Harvard, Boston -- US.}
\cvitem{2014}{\emph{ENAR}, Baltimore -- US.}
\cvitem{2014}{LCG 10 year anniversary, LCG-UNAM, Cuernavaca -- MX.}
\cvitem{2013}{\emph{useR2013}, Albacete -- Spain.}
\cvitem{2011}{\emph{BioC2011}, FHCRC, Seattle -- US.}
\cvitem{2010}{\emph{From Functional Genomics to Systems Biology}, EMBL, Heidelberg -- Germany.}
\cvitem{2010}{\emph{BioC2010}, FHCRC, Seattle -- US.}
\cvitem{2009}{\emph{BioC2009}, FHCRC, Seattle -- US.}
\cvitem{2009}{Course on Oral Communication taught by the master Rafael Popoca, CCG-UNAM, Cuernavaca -- MX.}
\cvitem{2008}{\emph{BioC2008}, FHCRC, Seattle -- US.}
\cvitem{2008}{\emph{A Short R/Bioconductor Course} by James Bullard from UC Berkeley, LCG-UNAM, Cuernavaca -- MX.}
\cvitem{2007}{\emph{Boston Bacterial Meeting}, Boston -- US.}
\cvitem{2007}{\emph{Retreat of the Department of Microbiology and Molecular Genetics - Harvard}, Boston -- US.}
\cvitem{2006}{\emph{Winter School in Genomics}, CCG-UNAM, Cuernavaca -- MX.}
\cvitem{2005}{\emph{HUGO 2005}, Kyoto -- Japan.}

\section{Software}
\subsection{Bioconductor}
\cvitem{2015}{bumphunter: contributor role.}
\cvitem{2014}{derfinder: Annotation-agnostic differential expression analysis of RNA-seq data at base-pair resolution -- main author.}
\cvitem{2014}{derfinderPlot: plotting functions for derfinder results -- main author.}
\cvitem{2014}{regionReport: generate HTML reports for exploring a set of regions -- main author.}
\cvitem{2014}{derfinderHelper: helper functions for derfinder package -- main author.}
\cvitem{2014}{derfinderData: data for derfinder examples -- main author.}
\cvitem{2014}{ballgown: contributor role.}


\subsection{Other R packages}
\cvitem{2014}{dots: simplify function calls \url{https://github.com/lcolladotor/dots}.}
\cvitem{2013}{fitbitR: visualize your FitBit data \url{https://github.com/lcolladotor/fitbitR}.}
\cvitem{2011}{BacterialTranscription: identify TSSs and TUs from RNA-seq data.}

\subsection{shiny applications}
\cvitem{2014--2015}{MPH capstone TA office hours sign up \url{https://github.com/lcolladotor/MPHcapstoneTA}}
\cvitem{2014}{Simple mortgage calculator \url{https://github.com/lcolladotor/mortgage}}

\section{Computer skills}
\cvitemwithcomment{all-purpose}{R}{Ranked 163/6361 in the US and 483/57692 worldwide by \httplink[GitHub Awards]{github-awards.com/users/lcolladotor} as of December 19, 2016. Does not take into account contributions at \httplink[LieberInstitute]{github.com/LieberInstitute} and \httplink[leekgroup]{github.com/leekgroup} organizations.}
\cvitemwithcomment{statistics}{Stata}{}
\cvitemwithcomment{scripting}{bash, Perl}{}
\cvitemwithcomment{markup}{LaTeX, markdown}{}
\cvitemwithcomment{OS}{Linux}{}
\cvitemwithcomment{cluster queue}{Sun Grid Engine}{}

\section{Languages}
\cvitem{Native}{Spanish}
\cvitem{Bilingual}{English}
\cvitem{Basic}{French}

\section{Other}
\cvitem{2012--2014}{Organized the \emph{Genomics for Students} group \url{http://www.biostat.jhsph.edu/~lcollado/GenomicsForStudents.html}}
\cvitem{2012--2014}{Organized \emph{Cultural Mixer} events for the Department of Biostatistics (JHSPH) with Amanda Mejia.}
\cvitem{2009--2011}{Organized a Genomics Journal Club at IBT-UNAM.}
\cvitem{2008--2009}{Elected class representative for the LCG Academic Committee.}
\cvitem{2008--2009}{Class representative for Administration Unit for Technology Information committee.}
\cvitem{2008}{Helped start the National Node of Bioinformatics online forum \url{http://foros.nnb.unam.mx/}.}


\end{document}


%% end of file `template.tex'.


